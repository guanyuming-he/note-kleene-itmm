\subsection{How do we come up with the proof}
In contrast to the polished proof, which isn't so intuitive, I investigated
into a rough way of how that can be arrived at.

Let us start with 
\begin{align*}
M \otoc^1 N_1 \subset N \\
N \otoc^2 M_1 \subset M
\end{align*}

Although both have a subset as one end, we can try to use one of them for the
subset and the other for the rest. That is, let $A_0 := M-M_1$, and set up

\paragraph{Rule 1}
For any $m \in M$,
\begin{enumerate}
	\item If $m \in M_1 = M-A_0$, then it corresponds to the $n$ that $n
		\otoc^2 m$.
	\item Otherwise ($m \in A_0$), it corresponds to the $n$ that $m \otoc^1
		n$.
\end{enumerate}

Now we ask, does Rule 1 establish a one-to-one correspondence? Unfortunately,
it does not. Let $m \in A_0$. It has a $n$ by case (2), i.e.\ $m \otoc^1 n$;
but for any $n$, we have anothre $m' \in M$ by $\otoc^2$. This $m'$ must be in
$M_1$ and thus $m \ne m'$. According to  Rule 1, $m$ correspond to $n$ by case
(2), while $m'$ correspond to $n$ by case (1). This shows that this rule does
not produce a 1-to-1.

