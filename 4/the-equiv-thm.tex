\subsection{How do we come up with the proof}
In contrast to the polished proof, which isn't so intuitive, I investigated
into a rough way of how that can be arrived at.

% one-to-one correspondence.
% make \sim a mathop so that with \limits its superscript goes above
% corro and corrt stands for correspondence one/two, respectively.
\def\corro{\mathop{\sim}\limits^1}
\def\corrt{\mathop{\sim}\limits^2}

Let us start with 
\begin{align*}
M \corro N_1 \subset N \\
N \corrt M_1 \subset M
\end{align*}

Although both have a subset as one end, we can try to use one of them for the
subset and the other for the rest. That is, let $A_0 := M-M_1$, and set up

\paragraph{Rule 1}
For any $m \in M$,
\begin{enumerate}
	\item If $m \in M_1 = M-A_0$, then it corresponds to the $n$ such that $n
		\corrt m$.
	\item Otherwise ($m \in A_0$), it corresponds to the $n$ such that $m
		\corro n$.
\end{enumerate}

Now we ask, does Rule 1 establish a one-to-one correspondence? Unfortunately,
it does not. Let $m \in A_0$. It has a $n$ by case~2, i.e.\ $m \corro n$;
but for any $n$, we have anothre $m' \in M$ by $\corrt$. This $m'$ must be in
$M_1$ and thus $m \ne m'$. According to  Rule 1, $m$ correspond to $n$ by
case~2 , while $m'$ correspond to $n$ by case~1. This shows that this rule does
not produce a 1-to-1.

To fight against this obstacle, consider if we can remove some influence of
case~1 ($\corrt$) so that it doesn't fill that entirety of $N$. 

Which ones of $M_1$ to remove? How about we remove all possible such $m'$s
mentioned earlier? Such $m'$s are produced by mapping $A_0$ to $N_1$ by $\corro$
and back to $M_1$ by $\corrt$.  Call it $A_1$. By definition, $A_1$ is
disjoint with $A_0$. Define the image of mapping $A_0$ to $N_1$ by $\corro$ as
$B_0$.

The updated case~1 and case~2 are:
\paragraph{Rule 2}
For any $m \in M$,
\begin{enumerate}
	\item If $m \in M-A_0-A_1$, then it corresponds to the $n$ such that $n
		\corrt m$.
	\item Otherwise ($m \in A_0+A_1$), it corresponds to the $n$ such that $m \corro
		n$.
\end{enumerate}

Does this eliminate the problem?  Let $m_0$ in $A_0+A_1$. We have $m_0 \corro
n_0 \corrt m_1$.  If $m_0$ is in $A_0$, then by definition, $m_1$ is in $A_1$.
For this $m_1$ we can't say too much, because it doesn't fall under case~1. But
what if $m_0$ is in $A_1$, or what if we continue digging from $m_1$?

We have $m_0 \corro n_0 \corrt m_1 \corro n_1 \corrt m_2$. 
\begin{enumerate}
\item Because $M_1$ is disjoint with $A_0$, This $m_2 \in M_1$ doesn't belong
	to $A_0$.
\item $m_1$ is in $A_1$, disjoint from $A_0$, the set of $m_0$s. Because
	$\corro$ is 1-to-1, it follows that the set of all $n_1$ must be disjoint
	from that of all $n_0$, $B_0$. Call the former $B_1$. Because $\corrt$ is
	one-to-one, the set of $m_2$'s, $A_2$ must be disjoint from that of
	$m_1$'s, $A_1$.  
\end{enumerate}

Thus, these $m_2$'s belong to neither $A_0$ nor $A_1$.  Because $m_1$ $\corro$
$n_1$, $m_1$ and $n_1$ are matched under the new rule by case~2. Because $n_1$
$\corrt$ $m_2$, they are also matched under the new rule by case~1. Again, this
new rule is still not a 1-to-1.

Continuing with this infinitely, and we will have the situation discussed in
the book.

